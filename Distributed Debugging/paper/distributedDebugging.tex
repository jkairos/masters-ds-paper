\documentclass[12pt]{article}

\usepackage{sbc-template}

\usepackage{graphicx,url}
\usepackage{listings}
\usepackage{subfigure}

\RequirePackage[vlined,titlenumbered,algo2e,ruled,portuguese]{algorithm2e} 

\usepackage[brazil]{babel}   
\usepackage[T1]{fontenc}
     
\sloppy

\title{Debugging Techniques for Distributed R-Trees}

\author{S�vio Teles\inst{1}, Jose Ferreira de S. Filho\inst{1}}

\address{Instituto de Inform�tica -- Universidade Federal de Goi�s (UFG) \\
  Alameda Palmeiras, Quadra D, C�mpus Samambaia \\
  131 - CEP 74001-970 -- Goi�nia -- GO -- Brazil
\email{savioteles@gmail.com, jkairos@gmail.com}
}

\begin{document} 

\maketitle

\begin{abstract}
  R-Trees arrived on the scene in 1984 by Antonin Guttman, as a mechanism of handling spatial data efficiently, building an index structure that helps retrieve data items quickly according to their spatial locations. The R-tree has found significant use in both theoretical and applied contexts. However, it brings a big challenge in terms of debugging.
  In this paper, we argue that distributed debugging is a non-trivial task and few efforts have been developed for distributed debugging of spatial objects based on R-Trees. We use examples of this issue and briefly discuss basic debugging techniques, then we propose RDebug, a technique for debugging a distributed R-Tree.       
\end{abstract}

\section{Introduction}
\label{sec:intro}

The increasing of large spatial datasets demands high performance engine in order to process complex spatial models. 
The best cost-benefit to provide innovative GIS applications taking advantage of all available data is through distributed and parallel GIS processing. 
But develop high performance engine to distributed spatial computing is very complex and challenging.

In order to handle spatial data efficiently, a database system needs an index mechanism that will help it retrieve data items quickly according to their spatial locations. 
The R-Tree typically is the preferred method for indexing spatial data. Many researches such as  \cite{an1999storing,dedsi,zhong2012towards}, 
show that a distributed index structure can provide an efficient mechanism of spatial operations processing.

However, distributed R-Trees indexes for Big Spatial Data are very complex to be developed and so it demands novel approaches to debug and check stability and this is the main issue investigated in this work.

Debugging is an essential step in the development process, though often neglected in the development of distributed applications 
due to the fact that distributed systems complicate the already difficult task of debugging \cite{cheung1990Framework}.
In recent years, researches have been developed some helpful debugging techniques for distributed environment. 
Nevertheless, we have not found any technique to debug distributed spatial indexes.

In this paper, we propose a new debugging algorithm for distributed R-Tree building. 
The debugging algorithm, called RDebug, uses the distributed index structure to aggregate debugging information. 
RDebug is used on DistGeo, a shared-nothing platform for distributed spatial algorithms processing. 
We also created a graphical tool to visualize the debugging information and the R-Tree index structure, called RDebug Visualizer. 

The main contributions of this paper are as follows:

\begin{itemize}
  \item RDebug - A debugging technique for distributed R-Tree building.
  \item DistGeo - A peer-to-peer platform, with no single point of failure, to process distributed spatial algorithms of an R-Tree.
  \item RDebug Visualizer - A graphical tool to visualize debugging information and the distributed R-Tree index.
\end{itemize}

The rest of the paper is structured as follows. In Section \ref{sec:related}, we briefly give an overview of the use of debugging techniques for distributed environments and the view of the distributed spatial algorithms. 
Section \ref{sec:spatial_dist} describes the distributed processing of spatial algorithms, 
Section \ref{sec:rdebug} presents our approach for distributed R-Tree debugging. Section \ref{sec:evaluation} presents the evaluation of RDebug algorithm in the DistGeo platform.
Finally, we close the paper with some concluding remarks in Section \ref{sec:conclusion}.

\section{Distributed Processing of Spatial Algorithms}
\label{sec:spatial_dist}		
A number of structures have been proposed for handling multi-dimensional spatial data, such as: 
KD-Tree \cite{bentley1975multidimensional}, Hilbert R-Tree \cite{kamel1994hilbert} and R-Tree \cite{guttman1984r}.
The R-Tree has been widely used to index the datasets on GIS databases and it has been used as an index data structure in this work.

An R-Tree is a height-balanced tree similar to a B-Tree \cite{comer1979ubiquitous} with index records in its leaf nodes containing pointers to data objects. 
The key idea of the data structure is to group nearby objects and represent them with their minimum bounding rectangle (MBR) in the next higher level of the tree. 

Figure \ref{fig:rtree} illustrates the hierarchical structure of an R-Tree with a root node, internal nodes ($N1...2 \subset N3...6$) and leaves ($N3...6 \subset a...h$). 
Every internal node contains a set of rectangles and pointers to the corresponding child node and every leaf node contains the rectangles of spatial objects.

The Figure \ref{fig:rtree-space} shows MBRs grouping spatial objects of $a...h$ in sets by their co-location. 
The Figure \ref{fig:rtree-index} illustrates the R-Tree representation. Each node stores at most $M$ and at least $m \leq M/2$ entries \cite{guttman1984r}. 
Our work uses the formula for $M$ value calculation presented in \cite{dedsi}.

\begin{figure}[h]
  \centering
  \subfigure[R-Tree index]
  {
  \includegraphics[width=0.45\textwidth]{rtree}
  \label{fig:rtree-index}
  } \qquad
  \subfigure[Geographic space]
  {
  \includegraphics[width=0.45\textwidth]{rtree-space}
  \label{fig:rtree-space}
  }
  \caption{R-Tree Structure}
  \label{fig:rtree}
\end{figure}

The Window Query is one of major query algorithms in R-Tree.
The search starts from the root node of the tree and the input is a search rectangle (Query box). 
For each rectangle in a node, it has to be decided whether it overlaps the search rectangle or not. If so, the corresponding child node has to be searched too. 

Searching is done recursively until all overlapping nodes have been traversed. 
When a leaf node is reached, the contained bounding boxes (rectangles) are tested against the search rectangle 
and the objects that intersects with the search rectangle are returned.

In Figure \ref{fig:rtree}, the search starts on root node, where the window intersects with nodes $N1$ and $N2$. Then, the algorithm analyses node $N1$, 
which only $N4$ intersects with the window. Analysing node $N4$, the algorithm returns the spatial object namely $'b'$, that is the single object that intersects the window.

In node $N2$, we do not have any entry intersecting with the window due to the dead space. 
In other words, the window intersects with a space, which does not contain any data.
The dead space should be minimized to improve the query performance, since decisions which paths have to be traversed can be taken on higher levels. 

The overlapping area between rectangles should be minimized as well, as it degrades the performance of R-Tree \cite{beckmann1990r}. 
Less overlapping reduces the amount of sub-trees accessed during r-tree traversal. The area between c and d in Figure \ref{fig:rtree} is an example of overlapping.

\subsection{DistGeo: A Platform of Distributed Spatial Operations for Geoprocessing}
\label{sub:dist_geo}	

DistGeo is a platform to process spatial operations in a cluster of computers (Figure \ref{fig:dist}). 
It is based on a shared-nothing architecture, which the nodes do not share CPU, hard disk and memory and the communication relies on message exchange. 
Figure \ref{fig: Figure 3} depicts DistGeo platform based on peer-to-peer model presented as a ring topology. 
It is divided in ranges of keys, which are managed for each server of the cluster. In order to a server join the ring it must be assigned a range first.

The range of keys are known by each server in the cluster using a Distributed Hash Table (DHT) to store the mapping of the keys to servers. 
For instance, in a ring representation, whose key set start with 0 to 100, if we have 4 nodes in the cluster, the division could be done as shown below: 
a) 0-25, b) 25-50, c) 50-75 e d) 75-100. If we want to search for one object with key 34, we certainly should look on the server 2.

\begin{figure}[h]
  \centering
  \subfigure[DistGeo Architecture]
  {
  \includegraphics[width=0.45\textwidth]{figure3.png}
  \label{fig: Figure 3}
  } \qquad
  \subfigure[R-Tree Partitioning in DistGeo]
  {
  \includegraphics[width=0.45\textwidth]{r-tree-partiotioning.png}
  \label{fig:partitioning}
  }
  \caption{DistGeo Platform}
  \label{fig:dist}
\end{figure}
	
Every replica of an object is equally important, in other words, there is not a master replica. Read and write operations may be performed in any server of the cluster. 
When a request is made to a cluster's server, it becomes the coordinator of the operation requested by the client. 
The coordinator works as a proxy between the client and the cluster servers. 
	
DistGeo uses the Gossip protocol \cite{demers1987epidemic}, which every cluster server exchanges information 
among themselves for everyone knows the status of each server. 
In the Gossip protocol every second a message is exchanged among three servers in the cluster, 
consequently every cluster's server have knowledge of each other. 

Figure \ref{fig:partitioning} illustrates the structure of a Distributed R-Tree in a cluster. 
The partitioning it is performed grouping the servers in cluster and creating the indexes according to the R-Tree structure. 
The lines in Figure \ref{fig:partitioning} show the need for message exchange to reach the sub-trees during the algorithm processing. 

Insertions and searching in a distributed R-Tree are similar to the non-distributed version, except for: i) The need of message exchange to access the distributed partitions and
ii) Concurrency control and consistency due to the parallel processing in the cluster. Both were implemented on DistGeo platform.

The distributed index has been built according to the taxonomy defined in \cite{an1999storing}, as follows: i) Allocation Unit: block - A partition is created for every R-Tree node; 
ii) Allocation Frequency: overflow - In the insert process, new partitions are created when a node in the tree needs to split; 
iii) Distribution Policy: balanced - To keep the tree balanced the partitions are distributed among the cluster servers.
	
Reliability and fault-tolerance were implemented on DistGeo storing the R-Tree nodes in multiple servers in the cluster. 
The DistGeo uses Apache Cassandra \cite{cassandra1apache} database to store the distributed R-Tree index nodes on cluster servers.
Each R-Tree node N receives a key, which is used to store the node in a server S responsible for ring range, replicating the node N to the next two servers in S (clockwise). 
If a message is sent to N, is selected one of the servers that store a replica of N.
The query requests are always sent to one of the cluster's server that stores the root node of the R-tree. 

As discussed on Section \ref{sec:spatial_dist}, reducing the overlapping and dead area on R-Tree minimizes the number of R-Tree nodes accessed during the tree traversal on search algorithms.
The growth of the number of nodes accessed increase the network traffic because the R-Tree nodes are stored in several servers on cluster, as shown in Figure \ref{fig:partitioning}.
This work implements a new algorithm that collects debugging information about a distribute R-Tree and can helps to reduce the overlapping and dead area.
We cover this algorithm in more details in Section \ref{sec:rdebug}.

\section{A Technique for Debugging A Distributed R-Tree}
\label{sec:rdebug}

	Spatial index debugging is a big challenge in a distributed R-Tree and this section describes a new technique RDebug, which allows debugging the index building of an R-Tree.
	
	The R-tree index building follows a top-down approach, in other words, the index is always built from root to leaves. Debugging the index reliability as the index is built is a non-trivial task, and the aim of this paper is to show a technique for index debugging after it has been created. Common challenges when working with an R-Tree are: i) Reliability of the nodes replicas of the R-Tree, ii) ensure that the MBR of the parent nodes intersect the MBR of their children, iii) the existence of duplicated nodes or being referenced by more than one parent node, and iv) if the value M and m of the nodes are compliant with the R-Tree descriptions as shown in Section \ref{sec:spatial_dist}. Furthermore, it is possible to access index data to help in its optimization as dead space and overlapping area.

	Algorithm \ref{alg:rdebug} shows the RDebug technique for debugging the distributed spatial index, using the index structure itself. The algorithm has two steps:
1) The algorithm processing is similar to the search in an R-Tree; 2) The algorithm does the inverse of a search in an R-Tree appending information to the distributed index.

	The first step, called S1 [Search sub-trees] (lines 1 - 11), the Algorithm \ref{alg:rdebug} traverses every node of the R-Tree starting from the root node to the leaves. Its purpose is to spread the debugging algorithm. The first request is sent to any server, which stores a replica of the root node.

	If the node $T$ is not a leaf (lines 2 - 8), then the number of children entries is stored to control the number of expected answers to this node in the second step of the algorithm. This information is stored in a shared memory accessed by all servers with a replica of $T$. Lines 4 -7, show that for every entry $E$ in the node, a message is sent (continuing step S1) to any server that holds a replica of the child node of $E$, carrying on the first step in the children nodes. If the node is a leaf, the second step, named S2 [Aggregation] is started.

	Second step aim (lines 12 - 41) is to aggregate the information used for future debugging. This step receives the debugging information of every child node of $T$. Therefore, for a given node $T$ with $n$ children, the second step is invoked $n$ times in the node $T$.

	The index itself is used to aggregate this information, the computational resources of the cluster helps improve the debugging information aggregation time. The index reverse structure allows, besides of spanning the aggregation information processing, build the debugging aggregation information, as one node of the R-Tree is responsible to aggregate only the information of its children. 

	Line 13 verifies the consistency of $T$ in the servers that store any replica of $T$. Line 14 verifies the consistency of $M$ and $m$. Lines 16 and 17 calculate the dead space and overlapping area for each node of the R-tree. Those information help the insertion algorithm designer analyze the quality of the built index. Beside this, lines 18 - 22, we have information of the MBRs of each node for every entry of the node. This information can be used as an input to a tool capable of visualizing the index of the R-Tree.
	
	If the aggregation step is being executed in the leaves, then if  $T$ is the root node (lines 24 - 26), the node information are sent to the client application. IF $T$ is not the root node, in line 27, the information are sent to the parent node of $T$. If the aggregation step is in an internal node (lines 29 - 43), the algorithm aggregates the information of the children nodes. Line 29, the algorithm receives the information sent by the child node. Line 30, verifies if the MBR of the entry that points to the child node is indeed the same MBR sent by the child node.
	
	The information of the children nodes are stored in a shared memory, with concurrency control, by the replicas of $T$. Hence, line 31, those information are accessed from the shared memory. Line 32, adds the data processed from lines 29 and 30. Line 33, acquires the number of children nodes that sent debugging information in the shared memory of replicas $T$. This information is stored in the variable $count$ and $count$ is decremented to let the other replicas know.
	
	If every node has sent the answer, the variable count then will hold the value 0 and lines 35-39 are processed. If $T$ is the root node, then the information are sent to the client application, otherwise, those information are sent to the parent node of $T$. If the variable $count$ is greater than 0, then the client information are stored in the shared memory.	
		
\medskip
\begin{center}
\begin{minipage}{1\textwidth}
\begin{algorithm2e}[H]
\SetAlFnt{\small\sf}
 \DontPrintSemicolon
 \LinesNumbered
\SetAlgoLined
 \BlankLine
 \Entrada{$T$ reference to root node of R-Tree $tree$}
 \Saida{Debugging information about distributed R-Tree $tree$}
 \BlankLine
	
 S1 [Search subtrees]

\eIf{$T$ is not leaf}{
  store the number of child entries in each replica server of T\;
	
	\For{each entry $E$ in $T$} {
		$server \leftarrow $ choose one server, randomically,  that store one replica of $E$\;
		send msg to $server$ to process the node child of $E$ on step S1\;
	}
}
{
  verifiy the consistency of $T$ in others replicas\;
	Invoke step S2 [Aggregation]\;
}

S2 [Aggregation]

$replica\_consistency \Leftarrow$ verifiy the consistency of $T$ in others replicas\;
$node\_consistency \Leftarrow$	verify the consistency of $M$ and $m$ values of  $T$\;
add in $informations$: $replica\_consistency$ and $node\_consistency$\;

$overlap \Leftarrow$ overlap area of $T$\;
$dead\_area \Leftarrow$ dead area of $T$\;
$bound \Leftarrow$ MBR of $T$\;
$list \Leftarrow \emptyset$\;

\For{for each entry $E$ in $T$}{
	add the $MBR$ of $E$ in $list$\;
}

\eIf{$T$ is leaf}{
 \If{$T$ is root}{
		send response with R-Tree nodes information to app client\;
	}
	{
		send msg with $informations$ to parent of $T$\;
	}
}
{
	  $entry\_info \Leftarrow$ information sent by child node\;
    $mbr\_consistent \Leftarrow$ verify if the bound of the child node is equal to bound of entry of T that points to this child\;
		$informations \Leftarrow$ the child information stored on shared memory by replicas of $T$\;
    add in $informations$: $entries\_info$, and $mbr\_consistent$\;
		
		$count \Leftarrow$ retrieve the number of entries child which not sent a debugging response and decrement by 1 unit\;
		
    \eIf{$count$ == 0}{
        \eIf{$T$ is root}{
           send response with $informations$ to app client\;
        }
				{
				   send msg with $informations$ to parent of $T$\;
				}    
		}
		{
			store $informations$ on shared memory\;
		}
            
}
\caption{$RDebug(T)$ 
\label {alg:rdebug}}
\end{algorithm2e}
\end{minipage}
\end{center}

The algorithm \ref{alg:rdebug} was implemented in the DistGeo platform to collect the debugging information of the built distributed R-tree. Those information are used in the platform to find out indexing issues and to optimize the R-tree index for searching. Figure 4.1, shows a graphical tool created to visualize the structure of the distributed R-Tree index, using as the input the information generated by the distributed debugging algorithm in DistGeo platform.

[Colocar a Figura aqui!!]

With the aid of RDebug \ref{alg:rdebug} algorithm, it is possible debug the searching algorithms of an R-Tree. E.g: The Window Query algorithm shown on Section \ref{sub:spatialdata}. To tweak RDebug to Window Query, it is only needed add an window query in the first step and gather the aggregation information of the accessed nodes. Whereas, the algorithms that access diverse R-Trees, such as Spatial Join, need a deep change, as the algorithms can go through different paths.

\section{Related work}
\label{sec:related}

Several techniques exist for debugging distributed, concurrent, and parallel programs. Although none of the existing approaches provide support for debugging a distributed R-Tree index \cite{manolopoulos2003rth,jacox2007spatial}. Along with our discussion, we highlight the techniques that influenced the design of RDebug, a new algorithm for debugging a distributed R-Tree index and a peer-to-peer platform to process distributed spatial algorithms.

Researches on distributed spatial data either show techniques to debug distributed applications in general or techniques for R-tree distributed processing. The Section \ref{dist_debug} shows the distributed debugging researches and \ref{spatialdist} describes researches of platforms for processing distributed spatial algorithms.
	
\subsection{Distributed Debugging Techniques}
\label{dist_debug}

In \cite{remeD2011} the author breaks down debuggers in two main families: log-based debuggers (also known as post-mortem debuggers) and breakpoint-based debuggers (also known as online debuggers). Log-based debuggers insert log statements in the code of the program to be able to generate a trace log during its execution. Breakpoint-based debuggers, on the other hand, execute the program in the debug mode that allows programs to pause/resume the program execution at certain points, inspect program state, and perform step-by-step execution. 

Several breakpoint-based debuggers have been designed for parallel programs using message passing communication including p2d2 \cite{p2d21996}, TotalView \cite{totalView2009}, and Amoeba \cite{amoeba1989}. These debuggers offer the traditional commands to, e.g. stop, inspect and step-by-step execution of a running program. Some of them allow to set breakpoints on statements of one process (e.g. TotalView) or a set of processes (e.g. p2d2, prism). 

An interesting alternative to traditional breakpoints is message breakpoints \cite{debuggingMP1997}. A message breakpoint stops all receiver processes of the next message sent by a process.The notion of message-breakpoints maps well to the event- loop concurrency model, since it allows to define sensible stepping semantics at message passing level.

A great body of concurrent and parallel debugging techniques are event-based. Event-based debuggers \cite{mcdowell1989Debugging} conceive the execution of a program as a sequence of events. The debugger records the history of the events generated by the application, which can then be used to either browse the events once the application is finished \cite{xtrace2007,causeway2009}, or to replay the execution to recreate the conditions under which the bug was observed. Event-based debuggers have been mainly criticized because the recording process is costly and browsing event history does not scale since manually inspecting huge traces becomes cumbersome and difficult.
 
\cite{cheung1990Framework} describes a process for distributed debugging in general and does not focus on a specific debugger or a particular technique, the paper focus is on defining a step by step approach to tackle distributed debugging independent of the environment. 
	
\subsection{Distributed Spatial Algorithms}
\label{spatialdist}

This Section describes briefly the researches that present the use of parallelism in order to improve the response time of the spatial algorithms. M-RTree \cite{koudas1996declustering} was the first published paper, which shows a shared-nothing architecture, with a master and several workstations connected to a LAN. The master machine handles a high volume of computation, besides of processing some directories of the R-Tree, it merges the answers to the client machines. Same technique is found on MC-RTree \cite{schnitzer1999master}.

In \cite{an1999storing}, it is described that a Network of Workstations (NOW) - also referred to as cluster systems  - is a cost-effective solution for high performance. The paper proposes an architecture similar to M-RTree and MC-RTree, with the same disadvantages of both.

In SD-RTree \cite{du2007sd}, a binary tree is used instead of an R-Tree. The binary tree increases the number of messages, since the data representation in a binary tree has more levels than the same data represented in an R-Tree. Hadoop-GIS \cite{kerr2009alternative} shows a scalable and high performance spatial data warehousing system for running large scale spatial queries on Hadoop. However the gain running large scale queries, it does not use indexing to improve the performance of the operations in the datasets.

\cite{dedsi} presents a distributed platform for spatial operations. Although, the solution proposed implements a distributed index, it is not scalable, since every message go through the replicated master node. [saviosbrc] shows a hybrid peer-to-peer platform to process the distributed spatial joint. The architecture presented in [saviosbrc] comprehends a set of machines for naming resolution and events management. Thus, these machines could be a bottleneck in the system.    

\cite{xie2008two} introduces a two-phase load-balancing scheme for the parallel GIS operations in distributed environment.

\cite{zhang2009spatial} describes MapReduce and shows how spatial queries can be naturally expressed in this model, without explicitly addressing any of the details of parallelization. Although the geocomputation high performance with this approach, it is only indicated for non-indexed datasets.

In \cite{zhong2012towards}, an approach is proposed for "indexing + MapReduce" data processing architecture to improve the computation capability of spatial query .The spatial index is distributed in two levels: locally and globally.

A number of techniques and platforms have been proposed for handling big spatial data, nevertheless  none of them propose a platform using a peer-to-peer approach for processing distributed spatial algorithms as found on DistGeo platform. Besides, none of the researches propose a technique for distributed spatial index debugging of an R-Tree.

\section{Conclusion}
\label{sec:conclusion}

A plataforma DistGeo apresenta uma solução completa para o processamento distribuído de operações
espaciais utilizando o índice R-Tree distribuído. Este processamento distribuído gera um desafio: como depurar o algoritmo de inserção em uma R-Tree com os nós distribuídos em um cluster de computadores. 

Nenhuma técnica de depuração dos algoritmos espaciais da R-Tree foi encontrado na literatura. Por isso, neste trabalho foi proposta uma nova técnica de depuração da construção do índice R-Tree distribuído denominado RDebug. Esta técnica utiliza o próprio índice R-Tree para coletar as informações de depuração. Esta coleta é realizada no índice R-Tree realizando o caminhamento na árvore de forma down-top. Utilizando a própria estrutura distribuída do índice, os dados podem ser coletados de forma distribuída, aumentando a eficiência no processamento do algoritmo de depuração.

Esta nova técnica foi implementada na plataforma DistGeo que possui arquitetura peer-to-peer. Os nós da R-Tree estão distribuídos e replicados pelo cluster de computadores. Por isso, o algoritmo RDebug pode ser processado sem que haja pontos de gargalo e pontos de falhas no cluster. Além disso, a replicação dos nós da R-Tree no cluster permitem que seja realizado um balanceamento de carga no caminhamento do índice distribuído R-Tree. Neste caminhamento, em cada acesso a um nó da R-Tree pode-se escolher a máquina com menor carga de processamento naquele momento, aumentando a eficiência do algoritmo RDebug. Estas informações do estado da máquina são trocadas utilizando o algoritmo Gossip.

Uma aplicação gráfica foi implementada para visualizar a estrutura do índice distribuído R-Tree e as informações de depuração da construção do índice. Com estas informações é possível identificar inconsistência na construção do índice e otimizar a qualidade do índice distribuído.

Em trabalhos futuros, o algoritmo RDebug será modificado para depurar de forma distribuída os algoritmos de busca Window Query e Join Query. O algoritmo RDebug pode ser adaptado facilmente para coletar informações de depuração da Window Query. Para o algoritmo Join Query ele deve sofrer diversas mudanças, já que o caminhamento é realizado em duas R-Trees distribuídas diferentes.
Serão realizados testes de desempenho para analisar a escalabilidade do algoritmo RDebug na plataforma DistGeo. 

\bibliographystyle{sbc}
\bibliography{sbc-template}

\end{document}