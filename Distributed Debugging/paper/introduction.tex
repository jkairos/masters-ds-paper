\section{Introduction}

Debugging is an essential step in the development process, albeit often neglected in the development of distributed applications due to the fact that distributed systems complicate the already difficult task of debugging.

Few techniques have been developed for debugging distributed spatial data; furthermore, no precise or elegant method has been developed for debugging distributed spatial data based on R-Trees data structures.

R-Trees are the most used data structures for spatial data access in a distributed environment, however searching algorithms and debugging is a non-trivial task as the R-Tree nodes are usually distributed in multiple nodes for efficient computation.

In recent years, researchers have been developed some helpful debugging techniques for distributed environment. Nevertheless, we have not found any technique to debug a distributed R-Tree.

The rest of the paper is structured as follows. In Section 2,  we briefly give an overview of the use of debugging techniques for distributed environments, Section 3 describes the distributed processing of spatial algorithms, Section 4 describes our approach for R-Tree debugging. Finally, we close the paper with some concluding remarks in Section 5.
