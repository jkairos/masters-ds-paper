\section{Evaluation}
\label{sec:evaluation}	

The RDebug algorithm have been evaluated on 3500 MHz Intel(R) Core(TM) i7-2600 CPU workstations
connected by 1 GBit/sec switched Ethernet running Ubuntu 14.04. Each node has 16 GB of main memory.
The experiment results were achieved with 1, 2, 4 and 8 servers on DistGeo platform.

The experiments used three data sets with different characteristics.
The first contains 1000000 points of business listings and points of interest (POIs) from SimpleGeo\footnote{https://github.com/simplegeo}.
The second dataset comprises 226964 lines representing the rivers on Brazil from LAPIG\footnote{www.lapig.iesa.ufg.br}. 
The third contains 220000 polygons of the census of USA from TIGER/Line\footnote{Census 2007 Tiger/Line data}.

The RDebug was validate on DistGeo platform after the insertion and indexing of each dataset.
The algorithm was able to collect information from the R-Tree index, such as dead area and overlap.
Furthermore, the RDebug algorithm has succeeded to collect the index structure allowing visualize each data set R-Tree index on RDebug Visualizer tool.

Three inconsistencies were inserted on purpose in the index to evaluate the RDebug: 
i) bound inconsistencies between parent and child nodes, ii) nodes filled with more than M entries and iii) duplication of a node on R-Tree. 
The RDebug algorithm was able to identify this inconsistencies on distributed R-Tree. 
In future works, node replica consistencies between servers will be simulated to evaluate the RDebug algorithm.