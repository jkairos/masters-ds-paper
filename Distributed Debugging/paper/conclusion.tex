\section{Conclusion}
\label{sec:conclusion}

DistGeo platform presents an approach for processing distributed spatial operations through the distributed R-Tree index. 
Due to the distributed processing nature on this platform an issue arises: debugging the R-Tree index distributed in a cluster of computers.

We have seen researches on spatial data processing and distributed debugging, but none of them propose techniques for debugging spatial algorithms in a R-Tree. 
Our work presents the RDebug algorithm for debugging the building of a distributed R-Tree index. 
RDebug uses the R-Tree index itself to gather the debug information. 
The data gathering is achieved in a distributed way, improving the debugging algorithm efficiency.

A new peer-to-peer platform (DistGeo) was proposed in our work to process distributed spatial algorithms.
RDebug has been implemented in DistGeo platform. The R-Tree nodes are distributed and replicated over the cluster. 
Thus, RDebug can be processed without bottlenecks and point of failures.

A graphical tool(RDebug Visualizer) has been created to visualize the structure of the distributed R-Tree index and the debugging information about the index building. 
Using this debugging information, we can identify discrepancies in the index building and optimize the R-Tree index too.
The RDebug algorithm can be used to collected debug information in any index with spatial nodes organization similar to R-Tree (e.g. Hilbert R-Tree \cite{kamel1994hilbert}).

Ongoing work includes modify the RDebug algorithm to debug the Window Query and Join Query searching algorithms. 
The RDebug algorithm is easily adapted to gather debugging information for Window Query. 
Whereas, for Join Query algorithm, RDebug must be changed considerably, since the traversal is processed in two different distributed R-Trees. 
Another ongoing work is to simulate node replica inconsistencies to evaluate the ability of the Rdebug to identify this inconsistencies.
On future works, the algorithm RDebug will be evaluated in larger clusters and performance results will be collected.