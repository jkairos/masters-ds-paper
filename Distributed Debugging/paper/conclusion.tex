\section{Conclusion}
\label{sec:conclusion}

DistGeo platform presents an approach for processing the distributed spatial operations through the distributed R-Tree index. 
Due to the distributed processing nature on this platform an issue arises: debugging the R-Tree index distributed in a cluster of computers.

We have seen researches on spatial data processing and distributed debugging, but none of them propose techniques for debugging the spatial algorithms of an R-Tree. 
In this paper, we present RDebug, a new technique for debugging the distributed index building of an R-Tree. RDebug, uses the R-Tree index itself to gather the debug information. 
The information gathering is done in the R-Tree index using a down-top traversal in the tree. 
Utilizing the distributed index itself, the data gathering can be achieve in a distributed way, improving the debugging algorithm efficiency.

RDebug, has been implemented in DistGeo platform. The R-Tree nodes are distributed and replicated over the cluster. Thus, RDebug can be processed without bottlenecks and point of failures. 
Besides, the R-Tree replicated nodes in the cluster allow load-balancing in the distributed R-Tree index traversal. 
During the traversal, at every node access of the R-Tree, the traversal might go to a node of the cluster with less workload increasing RDebug performance. 
The information exchange of the machines statuses is done trough the Gossip algorithm.

A graphical tool(RDebug Visualizer) has been created to visualize the structure of the distributed R-Tree index and the debugging information about the index building. 
With these input we can identify discrepancies in the index building and optimize it.

Ongoing work includes modify the RDebug algorithm to debug the Window Query and Join Query searching algorithms. 
The RDebug algorithm is easily adapted to gather debugging information for Window Query. 
Whereas, for Join Query algorithm, RDebug must be changed considerably, since the traversal is processed in two different distributed R-Trees. 
Another ongoing work is to validate the RDebug algorithm using datasets with different characteristics in DistGeo platform.