\section{Related work}
\label{sec:related}

Several techniques exist for debugging distributed, concurrent, and parallel programs. Although none of the existing approaches provide support for debugging a distributed R-Tree index \cite{manolopoulos2003rth,jacox2007spatial}. 
Along with our discussion, we highlight the techniques that influenced the design of RDebug, a new algorithm for debugging a distributed R-Tree index and a peer-to-peer platform to process distributed spatial algorithms.

Researches on distributed spatial data either show techniques to debug distributed applications in general or techniques for R-tree distributed processing. 
The Section \ref{dist_debug} shows the distributed debugging researches and \ref{spatialdist} describes researches of platforms for processing distributed spatial algorithms.
	
\subsection{Distributed Debugging Techniques}
\label{dist_debug}

In \cite{remeD2011} the author breaks down debuggers in two main families: log-based debuggers (also known as post-mortem debuggers) and breakpoint-based debuggers (also known as online debuggers). 
Log-based debuggers insert log statements in the code of the program to be able to generate a trace log during its execution. 
Breakpoint-based debuggers, on the other hand, execute the program in the debug mode that allows programs to pause/resume the program execution at certain points, inspect program state, and perform step-by-step execution. 

Several breakpoint-based debuggers have been designed for parallel programs using message passing communication including p2d2 \cite{p2d21996}, TotalView \cite{totalView2009}, and Amoeba \cite{amoeba1989}. 
These debuggers offer the traditional commands to, e.g. stop, inspect and step-by-step execution of a running program. 
Some of them allow to set breakpoints on statements of one process (e.g. TotalView) or a set of processes (e.g. p2d2, prism). 

An interesting alternative to traditional breakpoints is message breakpoints \cite{debuggingMP1997}. A message breakpoint stops all receiver processes of the next message sent by a process.
The notion of message-breakpoints maps well to the event- loop concurrency model, since it allows to define sensible stepping semantics at message passing level.

A great body of concurrent and parallel debugging techniques are event-based. Event-based debuggers \cite{mcdowell1989Debugging} conceive the execution of a program as a sequence of events. 
The debugger records the history of the events generated by the application, which can then be used to either browse the events once the application is finished \cite{xtrace2007,causeway2009}, 
or to replay the execution to recreate the conditions under which the bug was observed. 
Event-based debuggers have been mainly criticized because the recording process is costly and browsing event history does not scale since manually inspecting huge traces becomes cumbersome and difficult.
 
\cite{cheung1990Framework} describes a process for distributed debugging in general and does not focus on a specific debugger or a particular technique, 
the paper focus is on defining a step by step approach to tackle distributed debugging independent of the environment. 
	
\subsection{Distributed Spatial Algorithms}
\label{spatialdist}

This Section describes briefly the researches that present the use of parallelism in order to improve the response time of the spatial algorithms. M-RTree \cite{koudas1996declustering} was the first published paper, 
which shows a shared-nothing architecture, with a master and several workstations connected to a LAN. The master machine handles a high volume of computation, besides of processing some directories of the R-Tree, 
it merges the answers to the client machines. Same technique is found on MC-RTree \cite{schnitzer1999master}.

In \cite{an1999storing}, it is described that a Network of Workstations (NOW) - also referred to as cluster systems  - is a cost-effective solution for high performance. 
The paper proposes an architecture similar to M-RTree and MC-RTree, with the same disadvantages of both.

In SD-RTree \cite{du2007sd}, a binary tree is used instead of an R-Tree. The binary tree increases the number of messages, since the data representation in a binary tree has more levels than the same data represented in an R-Tree. 
Hadoop-GIS \cite{kerr2009alternative} shows a scalable and high performance spatial data warehousing system for running large scale spatial queries on Hadoop. 
However the gain running large scale queries, it does not use indexing to improve the performance of the operations in the datasets.

\cite{dedsi} presents a distributed platform for spatial operations. Although, the solution proposed implements a distributed index, it is not scalable, since every message go through the replicated master node. 
[saviosbrc] shows a hybrid peer-to-peer platform to process the distributed spatial joint. The architecture presented in \cite{de2013processamento} comprehends a set of machines for naming resolution and events management. 
Thus, these machines could be a bottleneck in the system.    

\cite{xie2008two} introduces a two-phase load-balancing scheme for the parallel GIS operations in distributed environment.

\cite{zhang2009spatial} describes MapReduce and shows how spatial queries can be naturally expressed in this model, without explicitly addressing any of the details of parallelization. 
Although the geocomputation high performance with this approach, it is only indicated for non-indexed datasets.

In \cite{zhong2012towards}, an approach is proposed for "indexing + MapReduce" data processing architecture to improve the computation capability of spatial query. 
The spatial index is distributed in two levels: locally and globally.

A number of techniques and platforms have been proposed for handling big spatial data, 
nevertheless none of them propose a platform using a peer-to-peer approach for processing distributed spatial algorithms as found on DistGeo platform. 
Besides, none of the researches propose a technique for distributed spatial index debugging of an R-Tree.